\documentclass[]{article}
\usepackage{amsmath}
\begin{document}

\title{Summary}
\author{Zi Yan}
\date{}
\maketitle

\section{Practical Network Support for IP Traceback}
We define the approximate trace- back problem as finding a candidate 
attack path for each attacker that contains the true attack path as a suffix. 
We call this the valid suffix of the candidate path.

All marking algorithms have two components: a marking procedure 
executed by routers in the network and a path reconstruction 
procedure implemented by the victim.

\subsection{Basic marking algorithms}
\begin{itemize}
 \item Node append: for any router $R$, append $R$ to each packet $w$
 
 Limitation: 1)high route overhead to append in flight, 2)length of the path is
 unknown, not enough space for record. 
 
 \item Node sampling: write router's address into a packet with probability
 $p$, only one record will remain. Because the router near the victim will
 have a bigger chance to mark a packet, therefore sorting the routers by number
 of marked packets will tend to give out the path.
 
 Limitations: 1)it is a slow process to reconstruct the path, and a large number
 of packets are needed, for far away routers have a less chance to mark a 
 packet. 2)not robust against multiple attacks.
 
 \item Edge sampling: two static fields reserved for start and end. If the router
 marks a packet, start is written and distance is set to 0, and if the router
 chooses not to mark, write $R$ to end when distance is 0, otherwise
 increase distance by 1. Distance prevent attackers from faking packets,
 if only a single attacker is there. Because the suffix, an attacker should
 forge a packet with longer or equal length of true attack path, which is 
 not possible.

\end{itemize}

\subsection{Encoding Issues}
To save per-packet space.
\begin{itemize}
 \item XOR the start and end IP addresses. Reconstruction will begin
 with the edge from the router adjacent to victim to the victim. A hop
 further router can be found by XOR again with a hop nearer router.
 \item only store part of bits of IP address and the offset in the IP address 
 of a route when marking. Cumulatively, the IP address will be recovered.
 \item Because part of IP cannot be unique, multiple attacks may have 
 a victim gets multiple edge fragments with the same offset and distance.
 combine IP and its hash value bit-interleavingly, and pick a fragment of
 this combination to mark, then the following router will XOR at the same 
 offset. 
\end{itemize}

\subsection{Reuse IP identification field}

\section{A Key-Management Scheme for Distributed Sensor Networks}
Choose part of keys in a key pool for each node. Shared keys can make
two sensors connected. Only need to make all sensors connected.

\subsection{Constraints}
\begin{itemize}
 \item Communication security constraints: encryption and decryption
 consume more power, if use asymmetric algorithm, than that uses 
 symmetric one. Transmission consumes more as well.
  \item Key management constraints: trusted third party is impractical.
  Single mission-key is unsafe, if one sensor is compromised. Pair-wise
  keys take too much space, and adding/removing is expensive. 
\end{itemize}

\subsection{Key distribution}
\begin{itemize}
 \item key pre-distribution: 1)generate a large pool of keys and their
 key identifiers, 2) randomly pick $k$ keys out, 3) loading the key ring
 into the memory of each sensor, 4) save key identifiers of a key ring
 and associated sensor identifier on a trusted controller node, 5) for 
 each node, loading i-th controller node with the key shared with that 
 node. 
 \item shared-key discovery: find shared keys from all neighbours by
 exchanging the key identifiers. Then the topology of all sensors can 
 be established by the connections between sensors.
 \item path-key establishment: path-key from the $k$ keys for two 
 sensors that are not  directly connected.
\end{itemize}

\subsection{Revocation}
Whenever a sensor is compromised, all keys in its key ring needs to 
be revoked. A controller node broadcasts list of revoked key identifiers.
The list is signed by $K^{ci}$ which is shared by the i-th controller with
each sensor at key pre-distribution phase.

After revocation, shared-key discovery and path-key establishment 
are needed for the disconnection due to the revoked keys.

Re-keying is like revoking a key from oneself. Shared-key discovery
and path-key establishment are needed.


\section{SPINS: Security Protocols for Sensor Networks}
We present a suite of security protocols optimized for sensor 
networks: SPINS. SPINS has two secure building blocks: SNEP and 
$\mu$TESLA. SNEP includes: data confidentiality, two-party data 
authentication, and evidence of data freshness. $\mu$TESLA provides 
authenticated broadcast for severely resource-constrained environments.

\subsection{Communication architecture}
The current prototype consists of nodes, small battery powered devices,
that communicate with a more powerful base station, which in turn is 
connected to an outside net- work.
 Base station: the root of a routing forest consisted of sensor nodes.
 
 Communication: 1) node to base station, 2) base station to node, 3) 
 base station to all nodes
\subsection{SPINS}
\begin{itemize}
 \item SNEP: data confidentiality, two-party data authentication, integrity,
 and freshness.
 \item $\mu$TESLA: authentication for data broadcast.
\end{itemize}

\subsubsection{SNEP}
Semantic security: 1)sender precedes the message with a random bit string.
2)Two counters shared by the parties (one for each direction of
 communication). 
 
 Use a message authentication code (MAC) to achieve two-party authentication
 and data integrity.
 
 How to get the keys from a master secret key $\mathcal{X}_{AB}$: To use
 the pseudo-random function $F$: encryption keys $K_{AB} = F_\mathcal{X}
 (1)$ and $K_{BA} = F_\mathcal{X}(3)$ for each direction of communication, 
 and MAC keys $K'_{AB} = F_\mathcal{X} (2)$ and $K'_{BA} = F_\mathcal{X}
 (4)$ for each direction of communication.
 
 The complete message that A sends to B is:
\begin{equation}
A \rightarrow B: \{D\}_{<K_{AB}, C_A>}, \text{MAC}(K'_{AB},C_A || \{D\}_{<K_{AB},
  C_A>})
\end{equation}
 where $D$ is data, $C$ is counter. 
 \begin{itemize}
  \item Semantic security. The counter makes the same message different 
  each time.
  \item Data authentication. MAC verifies correctly.
  \item Replay protection. The counter value in the MAC increment prevents
   replay.
  \item Weak freshness. A verified message will let the receiver know that
  the message is sent after previous verified message, because of the 
  increment of the counter value.
  \item Low communication overhead. The counter is kept at each point 
  without being sent in each message.
 \end{itemize}
Use nonce to get strong freshness, by means of sending request with a
nonce, and replying response with the MAC including the nonce.

Initialize a counter exchange:
\begin{align*}
A \rightarrow B:& C_A,\\
B \rightarrow B:& C_B, \text{MAC}(K'_{BA}, C_A||C_B)\\
A \rightarrow B:& \text{MAC}(K'_{AB}, C_A||C_B)
\end{align*}

Request current counter:
\begin{align*}
A \rightarrow B:& N_A,\\
B \rightarrow A:& C_B, \text{MAC}(K'_{BA}, N_A||C_B)
\end{align*}

To prevent DoS, when bogus messages are got many times, the nodes
can send the counter with messages. Another method is to attach another
short MAC to the message that does not depend on the counter.

\subsubsection{$\mu$TESLA}
Introduce asymmetry through a delayed disclosure of symmetric keys.
Beforehand, we need loose time synchronization. First, base station 
broadcasts a key that is secret at that time with MAC. Second, at the time
of disclosure, base station broadcasts the verification key.

$K_i = F(K_{i+1})$, F is one-way function. Use newer key to verify the old
keys. Once one key is disclosed, all keys precede it can be got by applying
$F$ on it.

$\mu$TESLA has multiple phases: 
\begin{itemize}
 \item sender setup. To generate a one-way key chain of length $n$, the 
 last key $K_n$, and computer all previous keys by using one-way function
 $F$.
 \item sending authenticated packets. Uniformed time intervals. Send message
 in time interval $i$ with $K_i$. In time interval $(i + \delta)$, reveal the key
 $K_i$.
 \item bootstrapping new receivers. Verify $K_{i+1}$ by $K_i = F(K_{i+1})$
 \item authenticating packets. Loose time sync for that the sender did not 
 yet disclose the key that was used to compute the MAC of an incoming
 packet. A new key $K_i$ can be authenticated with $K_v$ by using 
 $K_v = F^{i-v}(K_i)$.
 \item Nodes broadcasting authenticated data. Memory(store key chain) and
  energy(compute all keys from $K_n$) limit  a node broadcasting 
  authenticated data. Solution: 1) base station broadcasts
 instead of the node, 2) node broadcasts the data, but base station keeps
 one-way key chain and sends keys to the broadcasting node as needed.
\end{itemize}

\section{A Crawler-based Study of Spyware on the Web}
An automated solution to three problems:
\begin{itemize}
 \item Finding executables. 1) \textit{Content-type} HTTP header,
 2) URL contained an executable extension, 3) archives, to be extracted,
 4) JavaScript, find URL in script
 \item Running executables within a VM. simulate common user interaction,
 click next button, fill in user info.
 \item Analyzing the installed executable. Run anti-spyware tool in VM.
\end{itemize}

\textbf{Drive-by Downloads}: A drive-by download attack occurs 
when a victim visits a Web page that contains malicious content. 
To detect violation of the sandbox of an unmodified browser.

\section{Swift: A Fast Dynamic Packet Filter}
BPF: compilation, user–kernel copying, and security checking.

The primary objective of Swift is to achieve low filter update latency.
We attempt to avoid filter re-compilation and optimization, allow 
``in-place ” filter updating, and eliminate security checking.

ISA design can avoid compilation and security checking.

Two design choices are made to enable in-place filter modification: 
\begin{itemize}
 \item fixing instruction length. avoid the need to shift instructions
 on instruction replacement
 \item removing filter optimization. updates can be applied directly,
 and only updated part is copied from userspace to kernel. Use 
 hierarchical execution optimization instead.
\end{itemize}

\textbf{Hierarchical execution optimization}: Reuse existing instructions
as parents. Because new primitives usually for the same host but different
ports or the same protocol but different hosts. First evaluated parent Pass,
if succeeds, just halt, but if failed, execute child passes without re-executing
any copied instructions from parent.

\textbf{A Pass}: consisted of a series of instructions connected by logic
``AND". Those Passes are used independently and combined by logic
``OR". If all evaluations of One Pass are ``true", the packet is copied to
userspace. Otherwise, it will be evaluated by rest of Passes, and will be
dropped if failed all Passes.

\section{Packet Vaccine: Black-box Exploit Detection and 
Signature Generation}
\begin{itemize}
 \item white-box: need source code
 \item black-box: do not monitor a program's execution flow
 \item grey-box: no source code, but monitor execution flow
\end{itemize}

Rather than using expensive dataflow tracking, it detects and analyzes an
 exploit using the outputs of a vulnerable program.

A key step in most exploits is to inject a jump address to redirect the control
flow of a vulnerable program. Our approach is to check every 4-byte 
sequence (32-bit system) or 8-byte sequence (64-bit system) in a packet's
application payload, and then randomize those which fall in the address 
range of the potential jump targets in a protected program. 

\subsection{Address Range}
obtain a process's virtual memory layout.

\subsection{Vaccine Generation Algorithm}
\begin{itemize}
 \item  target address set contains
 stack address range, heap address range, and address ranges of 
 other objects, like global library functions.
 \item aggregate the application payloads of the packets in one 
 session into a dataflow. Find all byte sequence in target address 
 set.
 \item for each address found above, replace its most significant
 byte with a byte randomly picked from a scrambler set $R$ to get a
 new dataflow.
 \item construct vaccine packets with new dataflow.
\end{itemize}
$R$ should not contain symbols which could interrupt a protocol 
proceeding, like change "GET" in HTTP, and make the byte 
sequence go out of memory layout.

Exploit semantics should be preserved, and only the jump address
should be changed.

\subsection{Exploit Detection and Diagnosis}
To find exact address causes the segment fault, we can compare all
scrambled address with the exception log which includes the address
incur a exception. Again, we can change the found address, and
exploit the program again to see whether the program still throws
exception at the same point.

The CR2 and EIP registers are used to get exploit attempts.

\subsection{Signature Generation}
Use a known exploit as a template.

\subsection{Limitations}
1) may destroy exploit semantics while working on binary protocols.
2) cannot work on packets with encrypted payload or checksum.
3) signature have limited expression of exploit conditions.

\subsection{Architecture to protect Internet}
The found exploit signature will be added into packet filter, so an 
exploit will be caught at very first time, then be dropped by packet
filter.

\section{Phinding Phish: Evaluating Anti-Phishing Tools}
 Using our automated testing system, we were able to test how each 
 of 10 tools responded to a set of URLs multiple times over a 24 hour 
 period, allowing us to observe the effect of blacklist updates and 
 of phishing sites being taken down.

\subsection{Tool Exploits}
\begin{itemize}
 \item Content Distribution Networks: Making most blacklist-based
 tools not work, but SpoofGuard work better since it does not depend
 on URLs but a non-standard port on which website is on.
 \item Page Load Attack: A extreme long time to load a page will
 prevent tools which examine the content of a page, like SpoofGuard,
 from warning the user with red but not just yellow.
\end{itemize}


\section{Filtering Spam with Behavioral Blacklisting}

\subsection{IP-Based Blacklist}
\begin{itemize}
 \item Completeness: not all emails to spam traps are blacklisted, 
 false positive spamming emails do not get blacklisted in a short time.
 \item Responsiveness: taking too long to blacklist a spammer.
\end{itemize}

\subsection{Content-based}
Picture, pdf, or other non text attachments cannot be identified 
easily.

\subsection{Clustering algorithm}
First clustering, then classification. According to sending IP, received 
domains, and time. Time axis will be collapsed.

For classification, a recent behavior of an IP will be calculated a score,
which is the maximum value among all clusters similarity.

Use existing blacklist to bootstrap. 
 
\subsection{Design}
SpamTracker’s clustering algorithms rely on the assumption that the set 
of domains that each spammer targets is often more stable than the IP
 addresses of machines that the spammer uses to send the mail.
 
\begin{itemize}
 \item Clustering: 1) initial ``seed list" of bad IP, 2) behavior patterns
 for those IP. 
 \item Classification: give vector $r$ for one IP behavior, return score
 $S(r)$.
 \item Tracking Changes in Sending Patterns: In practice, re-clustering, 
 if behavior  cannot be classified, or say, map to any existing clusters. 
 In paper, re-clustering in a fixed time interval.
\end{itemize}

\subsection{Others}
Cannot find a single threshold to split spamming emails from normal
ones.

Can be improved by adding more feature for clustering, or give different
weight for different domains.

Incorporated with existing filtering systems or put in everywhere connected
to the Internet.

Only cluster rows are sent to compress data, and achieve better reliability
with replication and anycast.

Get IP behavior info from trusted source with secure channels. Or adjust
clustering time window.

\textit{SpamTracker} does better in across domain behavior analysis.

\end{document}


