\documentclass[12pt,letterpaper]{article}
\usepackage[utf8x]{inputenc}

\author{Zi Yan}
\title{CS6353 Assignment 3}
\date{}
\begin{document}

\maketitle

\section{Q1}
Solution:
\begin{itemize}
 \item[(a)] Because change other strings may interrupt the protocol,
 or destroy the exploit semantics. And most exploits are trying to 
 inject a jump address to redirect the control flow of a program.
 \item[(b)] Because Packet Vaccine does not look at the source code
 of a program, and never monitor on execution flow of a program, 
 but only the output of a program.
 \item[(c)] Because these attacks are mainly adopted by attackers,
 and they are taking advantage of injection of a jump address.
\end{itemize}

\section{Q2}
Solution:
\begin{itemize}
 \item[(a)] Once one sensor is compromised, all the communication
 channels will be insecure due to the key loss.
 \item[(b)] First, they broadcast all the identifiers of the keys they have.
 Second, they pick a key shared by both in their key ring according to
 the identifiers. An alternate method will let a node broadcast some 
 challenges encrypted by its keys, and the receiver can decrypt one by
 using a key shared by both, finally a shared key is found.
 \item[(c)] They will find a path between them, which consisted of nodes
 that can use shared keys to communicate, and then pick a path-key from
 unused keys in the key ring.
\end{itemize}

\section{Q3}
Solution:
\begin{itemize}
 \item[(a)] The author want to find out whether a single executable 
 contains a spyware or not, without other executables' interference.
 \item[(b)] Yes. If a scan shows a bunch of executables are clean, 
 those executables do not need to be scanned individually. Therefore,
 this way can save a lot of time.
 \item[(c)] Three times. 1) Group all eight in two groups, where each
 has four. 2) \textbf{Scan} one group. If scanned one is clean, separate 
 the other group into two groups, where each has two. Otherwise,
 separate scanned group into two, where each has two. 3) Pick one
 group to \textbf{scan}, and separate the not clean group as 2). 4) Now,
 we have two executables left. \textbf{Scanning} any of them can tell which
 one contains spyware.
\end{itemize}

\section{Q4}
Solution:
\begin{itemize}
 \item[(a)] 1) To save storage space by reducing node information (compared
 to node append), 2) to resist multiple attack paths (compared to node sampling).
 \item[(b)] To save storage space, because only one segment, offset information
 and distance will be stored. Because appending these information after a
  packet will be costly.
 \item[(c)] 1) For node sampling, it cannot reconstruct the path. 2) For edge
 sampling, if the routers far away from the victim have small probabilities to 
 mark a packet, the convergence of a path will be slow. When the far away
  routers have large probabilities to mark a packet, the convergence will be
  quick.
\end{itemize}

\section{Q5}
Solution:
\begin{itemize}
 \item[(a)] By using MAC of the message and the counter.
 \item[(b)] They need a master key $\mathcal{X}_{AB}$. For data encryption/decryption
 keys: $K_{AB} = F_\mathcal{X}(1)$, $K_{BA} = F_\mathcal{X}(3)$, and for MAC
 encryption/decryption keys: $K'_{AB} = F_\mathcal{X}(2)$, $K'_{BA} = F_\mathcal{X}(4)$.
 And $F_\mathcal{X}$ is a pseudo-random function.
 \item[(c)] The counter is authenticated by MAC, and it will increase monotonically 
 at each communication. So any replay attacks cannot succeed.
\end{itemize}

\end{document}