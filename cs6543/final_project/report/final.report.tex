\documentclass[12pt,letterpaper]{article}
\usepackage{url}
\usepackage[utf8x]{inputenc}
\usepackage{amsmath}

\author{Zi Yan}
\title{RFID Anti-Collision Protocols Summary}
\date{}
\begin{document}
\maketitle

\section{Introduction}
Radio Frequency Identification (RFID) systems are widely used in
our daily life, like payment by mobile phones, transportation
payments, and toll road payments\cite{wiki}. The systems can
identify objects remotely and wirelessly, so that they facilitate 
the situations that need identification of stuffs at the same time.
Therefore, more focuses are put on RFID, in order to make 
improvement on its performance.

In general, a RFID system has a reader and many tags. The reader
is typically a device reads information from those tags and is 
supplied with power, memory, and computational ability. And
the tags have various computational capabilities. When the 
passive tags can only give out information if the reader energizes
it, the active tags can send signals by themselves. The communication
distance also varies in accordance with whether the tag is passive or
not.

The most significant problem in RFID area is collision while a
reader is acquiring information from multiple tags. A sequential
order has to be enforced by all tags, thereby avoiding that two or
more tags may send their information at the same time. A collision
will make signals from two or more tags useless, leading to wastage
of bandwidth, energy, and increase of identification delays\cite{survey}. 

In \cite{survey}, three groups of anti-collision protocols are
discussed, which are Aloha-based protocols, tree-based protocols,
and hybrid protocols. There are still several variations in each
group. By using these protocols, a reader can identify a certain
amount of tags in a quick and less collisional way.

\section{Background} 
The RFID systems can be categorized in different ways, from 
communication method, operating frequency, to tag type.

\subsection{Communication Method}
Magnetic and electromagnetic coupling are the two ways of
identifying a tag. The former one differs from the latter one
in that an electromagnetic read can send signals further
than  a magnetic one. In other words, it is the difference
between near field and far field.

\subsection{Operating Frequency}
RFID systems use frequency band that ranges from 100KHz
to 5.8GHz\cite{survey}. 

\subsection{Tag Type}
There are passive tags, active tags, and ones between previous
two, called semi-passive tags. The difference among these tags
is how they are energized. Passive tags are energized by readers, 
whereas semi-passive and active tags have their own power source.
In addition, Passive, semi-passive, and active tags own computational
capabilities in a ascending order.

\section{Brief Introduction to Protocols}

\section{Framed Aloha Protocols}

\section{Discussion}

\section{Conclusion}

\bibliographystyle{plain}
\bibliography{ref} 
\end{document}