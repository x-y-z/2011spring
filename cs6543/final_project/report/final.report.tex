\documentclass[12pt,letterpaper]{article}
\usepackage{url}
\usepackage[utf8x]{inputenc}
\usepackage{amsmath}

\author{Zi Yan}
\title{RFID Anti-Collision Protocols Summary}
\date{}
\begin{document}
\maketitle

\section{Introduction}
Radio Frequency Identification (RFID) systems are widely used in
our daily life, like payment by mobile phones, transportation
payments, and toll road payments\cite{wiki}. The systems can
identify objects remotely and wirelessly, so that they facilitate 
the situations that need identification of stuffs at the same time.
Therefore, more focuses are put on RFID, in order to make 
improvement on its performance.

In general, a RFID system has a reader and many tags. The reader
is typically a device reads information from those tags and is 
supplied with power, memory, and computational ability. And
the tags have various computational capabilities. When the 
passive tags can only give out information if the reader energizes
it, the active tags can send signals by themselves. The communication
distance also varies in accordance with whether the tag is passive or
not.

The most significant problem in RFID area is collision while a
reader is acquiring information from multiple tags. A sequential
order has to be enforced by all tags, thereby avoiding that two or
more tags may send their information at the same time. A collision
will make signals from two or more tags useless, leading to wastage
of bandwidth, energy, and increase of identification delays\cite{survey}. 

In \cite{survey}, three groups of anti-collision protocols are
discussed, which are Aloha-based protocols, tree-based protocols,
and hybrid protocols. There are still several variations in each
group. By using these protocols, a reader can identify a certain
amount of tags in a quick and less collisional way.

\section{Background} 
The RFID systems can be categorized in different ways, from 
communication method, operating frequency, to tag type.

\subsection{Communication Method}
Magnetic and electromagnetic coupling are the two ways of
identifying a tag. The former one differs from the latter one
in that an electromagnetic read can send signals further
than  a magnetic one. In other words, it is the difference
between near field and far field.

\subsection{Operating Frequency}
RFID systems use frequency band that ranges from 100KHz
to 5.8GHz\cite{survey}. 

\subsection{Tag Type}
There are passive tags, active tags, and ones between previous
two, called semi-passive tags. The difference among these tags
is how they are energized. Passive tags are energized by readers, 
whereas semi-passive and active tags have their own power source.
In addition, Passive, semi-passive, and active tags own computational
capabilities in a ascending order.

\section{Brief Introduction to Protocols}
RFID systems adopt radio, via which readers can communicate with
and energize tags. There are mainly four different methods to share
the frequency band, i.e. space division multiple access (SDMA), 
frequency division multiple access (FDMA), code division multiple
access (CDMA), and time division multiple access (TDMA). Among
these four methods, TDMA is chosen by a large group of anti-collision
protocols, because it is simple and easy to implement.
\subsection{Aloha-based Protocols}
There are three variants of Aloha-based protocols, which are listed
as follows:
\begin{itemize}
 \item[1)] Pure Aloha (PA).
 \item[2)] Slotted Aloha (SA).
 \item[3)] Framed Slotted Aloha (FSA).
 	\begin{itemize}
 		\item[a)] Basic framed slotted Aloha (BFSA).
 		\item[b)] Dynamic framed slotted Aloha (DFSA).
 		\item[c)] Enhanced Dynamic framed slotted Aloha (EDFSA)
 	\end{itemize} 
\end{itemize}

For Pure Aloha, it is the most simple protocol. A tag, once it is
energized, will respond to the reader with its ID randomly. An
ACK or NAK will be replied to the tag, telling that it is identified 
or collided with others. A collision happens when two or more 
tags transit simultaneously, then collided tags need to back off.

For Slotted Aloha, it adds time slot, so that tags can only respond
within a time slot, and no partial collision will exist. Additionally,
synchronization has to be taken care of among tags and the reader.

A frame, consisted of several time slots, is introduced in Framed
Slotted Aloha, for the purpose of limiting the frequency tags 
transit within a frame. Each tag will transit only once within a
frame, lessening the chance of collision. We will look into one of
this kind of protocols in next section. Two more variants,
DFSA and EDFSA, are invented to adjust frame size dynamically
and group tags respectively. More efficiency is achieved by these
two variants.

Besides the basic variants above, there are several sub-variants
for each of them. 1) Muting will silence the identified tag, reducing
the number of active tags within a reader's interrogation zone.
2) Slow Down will reduce the identified tag's rate of  transmission. 
3) Fast Mode, on the other hand, silences tags except the one
is currently transmitting. 4) Early End is for SA and FSA, which
will end a time slot early if no transmission is carrying on.

\subsection{Tree-based Protocols}
Tree based protocols will allocate each tag a specific time period
to transit when all collisions are resolved. And muting is the standard
requirement for all tree based protocols. There are four categories:
\begin{itemize}
 \item[1)] Tree splitting (TS).
 \item[2)] Query tree (QT).
 \item[3)] Binary search (BS).
 \item[4)] Bitwise arbitration (BTA).
\end{itemize}

Tree splitting will split collided tags into $n$ groups, according to
the numbers generated by a random generator. Once each group
has only one tag, the algorithm will stop, and collision is eliminated.
Adaptive version will record the identifying order at each tag, then,
if no new tags are added or no one goes away, all the tags will remain
the same order. 

Query tree will let a reader to broadcast a prefix to all tags, once more
than one tag have the same prefix, namely a collision happens, the 
reader will append one more bit to the prefix until no collision is
there. I think it seems to re-assign IDs to each tag by using Huffman
code, because the query prefix varies in its length, which is the 
feature of Huffman code.

Binary search is simply like the one in the algorithm, sending a number
to all tags, then only tags below the number will respond. By doing so,
until only one tag responds, the tag is identified, and the reader will
repeat the above process to identify the rest tags.

The readers, that use Bitwise Arbitration, will request tags' IDs bit by bit,
until the tags reply with different bits at one time. The readers will silence
the tags respond with bit 1, and push the bit 1 into the stack it maintains,
after this, it continues to request until one tag is identified. The bits in
the stack will be popped out to activate silenced tags after active tags are 
identified, therefore previously silenced tags will be identified.

\subsection{Hybrid Protocols}
The hybrid protocols combine the aloha-based protocols with the tree-
based protocols, first grouping the tags with tree-based protocol, then
using aloha-based protocol to identify the tags in each group.


\section{Framed Aloha Protocols}
Framed Aloha is a variant of slotted Aloha, where within a frame, a tag
can transmit only once\cite{frame}. This scheme can significantly 
reduce the collisions caused by the same tag, because as the time 
goes by, fewer and fewer tags can transmit. If the frame size
is larger than the number of tags, the last tag will always be identified
unless it collides with the one before it. Compared to pure aloha and
slotted aloha, the total amount of transmissions from all the tags is
reduced, leading to fewer collisions.

\subsection{Passive and active systems}
For active tags, the reader will wake up all tags which are in sleep state, 
and collect responses from them. Then all identified tags will be put 
into sleep state, but the rest will continue to respond to the reader 
until all tags are identified. In addition, in each round, tags can only 
respond once. 

For passive tags, because they do not have any saved state information,
they will respond randomly but still only once in a frame. Unlike active
RFID, tag collisions may happen in every frame. As a result, in statistic,
we cannot 100 percent ensure all passive tags are identified, and we 
use an assurance level to stop the identification when this level is 
reached\cite{frame}.

\subsection{Mathematical Model}
In order to quantify the identification process, we use a statistic math
model to analyze the framed Aloha protocols. To simplify the analysis,
we assume tags are relatively static and no background noise is taken
into account. We only focus on the identification itself.
\subsubsection{Static frame size}
In \cite{frame}, a frame size is set to $N(i)$, and the number of tags is
$n(i)$, with respect to $i$th collection round. Then, the probability of
that $q$ tags are all in one slot is
\begin{equation}
p_q(i) = {n(i) \choose q} \left(\frac{1}{N(i)}\right)^q 
\left(\frac{N(i) - 1}{N(i)} \right)^{n(i) - q}.
\end{equation}
When $q$ is 1, it means only one tag appears in a slot, namely no 
collision happens and the tag is identified. Because this is binomial
distribution, therefore the expected number of successful identification
with in a frame is $N(i)p_1(i)$. The rest of analysis formulas can be
found in \cite{frame}.

From the analysis result\cite{frame}, we can find that as the frame size
increases, the expected collection rounds decreases. Because when 
the frame size is much larger than the number of tags, collisions will
be very rare, then fewer rounds are needed. Another important result
is that the increasing frame size will not always decrease the overall
tag read time. From figure 3(b) in \cite{frame}, the overall tag read
time sharply decrease when the frame size increases from 0 to the
area around the number of tags, and then gradually increase. 
This is because the limited frame size will cause a lot of collisions, 
but once the frame size can contain all the tags all together, the collision
will decrease very much. After that, the overall tag read time will be
extended due to the idle slots in a frame.

\subsubsection{Channel and capture effect}
In practice, the strongest or the earliest signal can be captured by 
the reader. Therefore, more tags can be identified even the collision
number is the same, compared with the previous math models.

\subsubsection{Dynamic frame size}
In application of the RFID systems, the number of the tags is unknown,
and a fixed size frame cannot be suitable for different sizes of tags.
Therefore, it is better to estimate the size of the tags that are in 
the interrogation zone.

In \cite{frame}, it guesses the number of tags from a constant value 
for garbled slots and a constant mapping from the tag set to the 
frame size. 

Additionally, in \cite{survey}, seven different tag estimation functions,
including the one in \cite{frame}, are introduced. They are using two
methodologies. The first is that computing the number by using a 
constant multiplier, like \cite{frame}. The other uses probabilistic
or statistical methods. 

The tag estimation can be done statically or dynamically. But static
ones perform better in a low tag range, while dynamic ones
gain more accuracy in a large tag range.

\subsection{Summary}
The framed slotted Aloha protocols reduce collision by limiting tags
to respond only once in one frame. Ideally, when the number of tags
is equal to the frame size, the system will gain the greatest throughput.
However, as discussed above, the overall tag read time attracts more
attention. From the analysis, the frame sizes for passive tags need to 
be larger than the ideal frame size, such that the least reading time
can be achieved. On the other hand, smaller frame sizes than the ideal
frame size will help active tags gain the least reading time. 

Another important factor is capture effect, due to which the reader 
will amplify the stronger signal as well as suppressing the weaker 
one. This effect can facilitate the tag identification, because even
at the situation that two tags cause a collision, the reader can still
identify one of them if the signal strengths differ from one to 
the other to some extent. Consequently, the reading time will be
reduced. 
\section{Discussion}
Aloha-based protocols and tree-based protocols represent probabilistic
and deterministic methods respectively. For probabilistic methods, it is 
simple and can dynamically adapt to the change on tags. On the other 
hand, deterministic methods will identify tags more accurate and with
high confidence, but require restarting reading process if new tags are
added. 

For hybrid protocols, they combine the advantages from both tree and
Aloha protocols. They either manipulate tree protocols' determinism at
the identification phase, such that the identified tags will remember
the slot number, or deploy the flexibility of Aloha protocols by using
Aloha at identification phase, reducing restarting reading times in a
tree protocol.

Moreover, the protocols introduced in \cite{survey} all focus on the 
immobile tags. Once tags are moving all the time, the performance
of those protocols will be quite different. Unlike \cite{frame}, in
\cite{survey}, the authors only consider the ideal situation, despite
capture effect. So it may be interesting to see the performance 
when capture effect is taken into account.

\section{Conclusion}
In \cite{survey}, we go through 31 kinds of tree protocols and 42
Aloha protocols, therefore we have a brief idea about RFID anti-collision
protocols. In \cite{frame}, we study the framed slotted Aloha 
in detail, and learn more about other factors that can affect the 
performance of framed slotted protocols. 

Consequently, for anti-collision protocols, how to identify tags 
quickly and with more confidence, how to save energy, and 
how to use less resource will be the main focus for future study.
\bibliographystyle{plain}
\bibliography{ref} 
\end{document}